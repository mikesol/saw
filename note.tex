%!TEX encoding = UTF-8 Unicode
\documentclass{article}
\usepackage{verbatim}
\usepackage{lmodern,textcomp}
\usepackage[utf8]{inputenc}
\usepackage[frenchb]{babel}
\usepackage[T1]{fontenc}
\usepackage{romande}
\usepackage{parskip}
\usepackage{longtable}
\usepackage{arydshln}
\usepackage{hyperref}
\title{Note d'intention de \emph{Sit Ozfårs Wysr}}
\author{Mike Solomon}
\date{}
\begin{document}
\thispagestyle{empty}
\maketitle
\thispagestyle{empty}
\emph{Sit Ozfårs Wysr} propose une réinterprétation polyglotte
et féministe du film \emph{Le Magicien d'Oz}.\par

Le spectacle fait basculer le rapport de force central de l'histoire
d'origine où la petite Dorothée,
distraite et perdue dans le bois, se fait aider par plusieurs hommes
pour rentrer chez elle. Or, dans \emph{Sit Ozfårs Wysr},
Dorothée, toujours une petite fillette aux nattes, est confiante et sans pitié dans sa quête de voir le magicien.
Le rapport de force central du
spectacle s'effectue entre elle et son chien Toto,
joué par un chanteur légèrement vêtu de cuir que Dorothée tient en laisse.
Tout au long du spectacle, Dorothée se sert des gens qu'elle rencontre
pour parvenir à ses fins.\par

Le choix d'écrire le spectacle en \emph{frizn}, langue
scandinave fictive, permet d'interpeller les spectateurs francophones
et anglophones de manière différente.
De \og{}gros mots\fg{} en français et en anglais parsèment le frizn dans des
contextes inattendus et innocents. Les longueurs et les sonorités
des mots par
rapport aux mots jumeaux en français et en anglais sont souvent absurdes. Les
sous-titres, qui n'ont pas forcément de rapport avec l'action sur scène,
juxtaposent la sobriété du frizn, la verbosité du français et
la culture des \emph{buzzwords} d'anglo-américains.
Enfin, frizn permet de créer une prosodie musicale à l'intérieur de la langue
qui correspond au monde sonore des chansons.\par

Dans la grande tradition française de l'opéra pastiche, \emph{Sit Ozfårs
Wysr} déconstruit une musique
d'origine pour en créer une nouvelle. Ce faisant, il apporte un regard
critique à la fois sur la musique et sur les procédures contemporaines utilisées
pour la dénaturer. Ce phénomène de \og{}tuer le père\fg{} musical
passe par le \emph{free jazz} (Dong Dong Sit Bish be Tost), l'échantillonnage (Ubr
sit Drizl-$\pi$r) ou bien la réharmonisation radicale (Sit Wundrful Ozfårs
Wysr).\par

\emph{Sit Ozfårs Wysr} est surtout jubilatoire. Il est conçu pour les cinq
voix de l'ensemble 101 (\url{http://www.ensemble101.fr}) et, comme tous les morceaux que
j'écris pour cette formation,
la musique et le livret de \emph{Sit Ozfårs
Wysr} débordent de joie et défendent une vision profondément optimiste, quoique
déjantée et tordue, de notre époque.
\end{document}
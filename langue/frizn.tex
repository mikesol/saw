%!TEX encoding = UTF-8 Unicode
\documentclass[a4paper]{article}
\usepackage[french]{babel}
\usepackage{multirow}
\usepackage{hyperref}
\usepackage{lmodern,textcomp}
\newenvironment{mstabular}{\begin{center}\begin{tabular}}{\end{tabular}\end{center}}
\usepackage{romande}
\usepackage[utf8]{inputenc}
\usepackage[T1]{fontenc}
\usepackage{longtable}
\usepackage{parskip}
\usepackage{graphicx}
\usepackage{pdfpages}


\usepackage[text={5.5in,10in},centering]{geometry}
\newcommand{\mschapter}[1]{\chapter{#1}} 
\newcommand{\mssection}[1]{\section{#1}} 
\newcommand{\mssubsection}[1]{\subsection{#1}} 
%\titlehead{\centering\includegraphics{main_graphic.png}}
\title{\emph{Frizn}\\Mode d'emploi}
\author{}
\date{}

\begin{document}
\maketitle
\section{Principes de construction}
\begin{itemize}
\item Que les principes sont un peu différent dans chaque scène.
\item Que les décalages se déclinent en fonction de la scène.
\item Que tout reste dans une optique locale toujours.
\end{itemize}

\section{Règles de fond}
\begin{itemize}
\item la longueur des voyelles et des consonnes varient beaucoup.

\item pluriel = fois deux. petitepetite fillefille rougerouge.
\item le pluriel se décline toujours en fonction du nombre.
\item si c'est très nombreux, quatre plus un geste.

\item ``fårs'' est omniprésent
\item tout est un point d'intérrogation

\item quelques expressions qui se servent toujours d'un concept de base qui
est traduit différement et qui commence par ``comme on dit''

\item \emph{Drisig} et un nom commun.
\end{itemize}

\section{Jeux}
\begin{itemize}
\item même phrase plusieurs fois de suite veut dire des choses différentes.
on fait l'inverse aussi - plein de choses différentes qui veulent dire la
même chose (qui s'enchaînent)
\item Traduction littérale d'un mot aggloméré.
\item associer des gestes à certaines façons radicales de parler (que des
consonnes, que des voyelles, que de l'air).
\item ``J'ai pas compris'' - soit un motif, soit singulier
\item 1 très longue phrase qui veut dire quelque chose de court, pareil dans
l'autre sens

\item courte respiration qui veut dire quelque chose
\item geste d'hésitation un peu maladroit

\item Eudes peut rire à un moment donné à quelque chose qui n'est pas du
tout drôle

\item Livre de traduction pour parler français à la fin.

\end{itemize}

\section{Gestes}
\begin{itemize}
\item Ne pas regarder l'autre quand on s'addresse à lui
\item Des façons de mettre la main sur la tête
\item Toucher des zones sensible du corps
\end{itemize}

\section{Sous-titres}
\begin{itemize}
\item Dire ``Expression frizn intraduisible'' lorsque l'autre l'explique
bien.
\item quelque chose 
\end{itemize}

\section{Jeu d'acteur}
\begin{itemize}
\item quelque chose de banal qui fait rire tout le monde
\end{itemize}

\end{document}
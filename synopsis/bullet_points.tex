%!TEX encoding = UTF-8 Unicode
\documentclass[a4paper]{article}
\usepackage[french]{babel}
\usepackage{multirow}
\usepackage{hyperref}
\usepackage{lmodern,textcomp}
\newenvironment{mstabular}{\begin{center}\begin{tabular}}{\end{tabular}\end{center}}
\usepackage{romande}
\usepackage[utf8]{inputenc}
\usepackage[T1]{fontenc}
\usepackage{longtable}
\usepackage{parskip}
\usepackage{graphicx}
\usepackage{pdfpages}


\usepackage[text={5.5in,10in},centering]{geometry}
\newcommand{\mschapter}[1]{\chapter{#1}} 
\newcommand{\mssection}[1]{\section{#1}} 
\newcommand{\mssubsection}[1]{\subsection{#1}} 
%\titlehead{\centering\includegraphics{main_graphic.png}}
\title{\emph{Sit Ozfårs Wysr} \\ Synopsis}
\author{}
\date{}

\begin{document}
\maketitle
\mssection{Sit Friznfårs Begrbegr}
\begin{itemize}
\item Moment d'accueil de la troupe nationale de Frizngård (en costumes
traditionnels) par les techniciens du théâtre.
\item Hymne national (que des sons de percu).
\item Donner un cadre qui sera également établi par d'autres éléments comme
le programme, des affiches à l'entrée (?). Cela permettra d'installer une
convention claire : les spectateurs vont assister à une adaptation du
Magicien d'Oz peu conventionnelle dans le style national frizn.
\item La scène et les costumes seront dans des tons et des matières froids
(miroir, paillette, carbo-glace, blanc, gris, bleuté, vert).
\item Entrée où on chante ``The Wonderful Wizard of Oz'' -- la première de
six variations très différentes du thème de base.
\end{itemize}
\mssection{Dörty et son chien Totö}
\begin{itemize}
\item ce sont ces deux personnages qui guident l'histoire
\item ce sont eux qui subissent et qui commentent les rencontres
\item bout d'écriture de la première scène
\item Somewhere Over the Rainbow
\item tornade (idée scénique -- la tornade sera évoquée par des
ventilateurs, des artifices - que ce soit pas cheap, mais plutôt comme un
tournage de film et surtout dans le respect total de la tradition
\emph{frizne} de scénographie)
\item trouver une équivalence scénique à la colorisation du film en arrivant
dans le pays d'Oz.
\end{itemize}
\mssection{Les sorcières}
\begin{itemize}
\item après être arrivés en Oz, on voit d'abord la bonne sorcière
\item les deux sorcières sont jouées par la même personne
\item elles représentent le bien et le mal dans un univers parallèle
\item elles pourraient être des speakerines de télévision
\item bout d'écriture
\item Le discours éclaté des sorcières intervient régulièrement
au cours de la pièce. Elles ne sont pas moteur de l'action mais représentent
une morale absurde qui intervient pendant des moment imprévisibles.
\item Munchkinland Medley - c'est un mélange de styles de jazz qui combinent
tous les morceaux de cette partie du film (Ding Dong the Witch is Dead,
Follow the Yellow Brick Road). La scénographie fait reférence
aux lilliputiens du film.
\end{itemize}
\mssection{Les trois amis}
\begin{itemize}
\item Leurs interventions sont plus physiques qu'intellectuelles. Ils ne
parlent pas.
\item Les trois amis sont joués par une même personne.
\item C'est Dörty et Totö qui observent les numéros successifs de
l'épouvantail, du bucheron et du lion. Ils projètent leur propres problèmes
(bêtise, manque de cœur, lâchété) sur eux.
\item bout d'écriture
\item Il y a trois chansons - If I Only Had a Brain, Heart, Nerve, qui font
reférence aux trois chansons du film et qui se terminent toujours avec une
reprise de The Wonderful Wizard of Oz.
\end{itemize}
\mssection{Grönland}
\begin{itemize}
\item Long crescendo musical mélangeant plusieurs morceaux du film
(Optimistic Voices, Merry Old Land of Oz, If I Were King of the Forest).
\item Montée d'intensité vers le moment critique de la rencontre avec le
Magicien.
\end{itemize}
\mssection{Sit Ozfårs Wysr}
\begin{itemize}
\item Personnage décadent, diabolique, dense, inquiétant, désinvolte, jeune,
beau, hypnotique.
\item Il apparait à la fin du spectacle et demande que l'on tue une
sorcière.
\item Désemparés, les personnages chantent The Jitterbug : reférence au
morceau swing qui a été supprimé du film.
\item A la fin, Dörty est sur le point de tuer la sorcière mais elle
explique qu'elle ne le fera pas parce que ce serait la fin de sa quête et
qu'elle préfère continuer le voyage. Tuer, c'est pas beau. La vie est belle.
Longue est la route, mais il faut profiter de chaque instant.
\item Nous nous amuserons à accentuer au maximum le décalage entre la langue
frizne et les traductions en anglais et en français.
\item bout de l'écriture
\item Pendant ce monologue, elle déclenche par inadvertance une série
d'évènements qui aboutissent tout de même à la mort accidentelle de la sorcière.
\end{itemize}
\end{document}
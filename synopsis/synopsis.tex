%!TEX encoding = UTF-8 Unicode
\documentclass[a4paper]{article}
\usepackage[french]{babel}
%\usepackage{polyglossia}
%\usepackage{bidi}
\usepackage{rotating}
%\newcommand{\mshrule}{\hrule}
\newcommand{\mshrule}{}
\usepackage{dashrule}

\usepackage{verbatim}
\usepackage{array}
\usepackage{multirow}
\usepackage{hyperref}
\usepackage{lmodern,textcomp}
\usepackage{romande}
\usepackage[utf8]{inputenc}
\usepackage[T1]{fontenc}
\usepackage{longtable}
\usepackage{parskip}
\usepackage{graphicx}
\usepackage{pdfpages}


\usepackage[text={160mm,247mm},centering]{geometry}
\newcommand{\mschapter}[1]{\chapter{#1}} 
\newcommand{\mssection}[1]{\section{#1}} 
\newcommand{\mssubsection}[1]{\subsection{#1}} 
%\titlehead{\centering\includegraphics{main_graphic.png}}
\title{\emph{Sit Ozfårs Wysr} \\ Synopsis}
\author{}
\date{}

\begin{document}
\maketitle
Sit Ozfårs Wysr, pièce de théâtre musical en un acte, est une adaptation du Magicien d'Oz
présentée par la Troupe Nationale de Frizngård, pays scandinave fictif.  A
travers les six scènes du spectacle, la pièce raconte à la fois l'histoire du Magicien
d'Oz et la culture étrangère et étrange de ce petit pays de 30 000
habitants. Le dialogue, épars, absurde et joyeux, relie des morceaux
musicaux qui s'inspirent de la musique du film \emph{Le Magicien d'Oz} de
1939 pour explorer de nouveaux regards sur le jazz et le chant lyrique. Le
livret, écrit en langue \emph{frizne}, est sous-titré en français et en
anglais.
\mssection{Sitsit FRIZN\-GÅRD\-FÅRS-FRIZN\-GÅRD\-FÅRS Begrbegr}
La pièce commence avec un bref accueil de la Troupe Nationale de Frizngård
(Sitsit FRIZN\-GÅRD\-FÅRS-FRIZN\-GÅRD\-FÅRS Begrbegr) par les techniciens du
théâtre.
Hormis les consignes habituelles (éteindre les portables, etc.), les
techniciens
expliquent que c'est le premier voyage à l'étranger de la troupe, qu'ils ne
parlent pas un mot de français et qu'il fait très chaud ici. Il jouent
le hymne \emph{frizn}, un morceau entièrement joué en se tapant sur la
joue. L'accueil des artistes étrangers sera établi également
par des éléments annexes présentés avant le spectacle comme:
\begin{enumerate}
\item Des petits drapeaux frizns à la vente. Le drapeau frizn est le seul
drapeau au monde qui est complètement transparent;
\item L’alcool frizn est offert à chaque spectateur. L'alcool frizn est une bouteille
faite entièrement de glace avec une frite à l'intérieur. Le processus de
fermentation date
d'avant l'âge de glace, où la frite interagissait avec l'eau fondante pour
déclencher la fermentation. Depuis l'âge de glace, la frite reste dans un
état gelé. Une grande partie de la religion frizne est consacrée aux prières
pour le Grand
Dégel, qui permettrait à l'alcool frizn de monter de 0\% à l'ancien
niveau de 0,5\%;
\item Un petit livre sur l'Histoire frizne. Le livre survole la monarchie \emph{frizne}
dont la princesse de Frizngård, Iik Drisig, est la seule représentante
vivante. Abl Drisig, qui est à la fois le frère, père et oncle de la princesse, est le seul
représentant mort de la ligne royale. Le livre discute également de la résistance républicaine
\emph{frizne}, cellule militante et terroriste qui lute contre le pouvoir
monarchique. Le chef de l'opposition est la princesse de Frizngård.
\item Un dictionnaire frizn-français, disponible à la vente. Le dictionnaire
ne reprend que quelques mots clés du spectacle Sit Ozfårs Wysr avec des
hypothèses de traduction proposés.
  \begin{itemize}
  \item Sit | le/la/les/un/une/des/gastronomie/tromper
  \item -fårs | mon/ma/ton/ta/son/à la/terme générique qui peut tout vouloir
  dire sauf ``sa'' et ``au''
  \item Wysr | magicien/sa/au
  \item Gode | le bien ou le mal (pas vérifiable)
  \item Notti | le mal
  \item grön\-pysin\-blad\-utsid\-smal\-fårs\-klmb\-sperm\-lo\-svit\-fårs\-gry\-plant\-qvik\-fårs\-anorexik\-sizr\-fårs\-fart\-fårs\-\_\-fårs\-fårs
  | l'herbe
  \item Drisig | nom propre
  \item frizn | le Nord/le Nord-Pas de Calais/éventuellement le Nord-Pas de
  Calais Picardie avec la reforme territoriale proposée par le Sénat en
  octobre 2014.
  \end{itemize}
\end{enumerate} \par
Après l'introduction, la troupe chante une réharmonisation
radicale de ``The Wonderful Wizard of Oz:
\href{http://youtu.be/atSEB4qxspM}{youtu.be/atSEB4qxspM}.
\begin{center}
\includegraphics[width=\textwidth]{ex1}
\end{center}
\mssection{Dörty et son chien Ttö}
Dörty et Ttö (Dorothy et Toto dans le conte d'origine) sont les deux
personnages qui guident l'histoire. A travers une série de rencontres avec
des sorcières, des amis improbables et le Magicien d'Oz, ils apprennent à
mieux se connaître et à dépasser la barrière des espèces. Ils sont
également les représentants principaux de la langue \emph{frizne}. Grâce à
leurs dialogues, on entrevoit les particularités de la langue et la futilité des
traductions.
\mshrule
\subsection*{Dörty}
\begin{center}
\begin{tabular}{|p{0.15\textwidth}|p{0.45\textwidth}|p{0.3\textwidth}|}\hline
Frizn (parlé) & Français (sous-titres) & Anglais (sous-titres) \\\hline
Sit by, Ttö? &
Regarde l'arc-en-ciel, Toto. Il brille de mille feux multicolores. Ici, tout est si
plat, si laid, si vieux, si gris, si terne, enfin inqualifiable par des
mots. Ressens-tu cela, Toto~?! Toi, un pauvre chien dont les capacités
affectives sont limitées par une défaillance d'hygiène.
Un jour, on ira là-haut ensemble~!! &
That's a rainbow.  We should probably go there.\\\hline
\end{tabular}
\end{center}\par
\subsection*{Ttö}
\begin{center}
\begin{tabular}{|p{0.5\textwidth}|p{0.2\textwidth}|p{0.2\textwidth}|}\hline
Frizn (parlé) & Français (sous-titres) & Anglais (sous-titres) \\\hline
Ttö lick sit grön\-pysin\-blad\-utsid\-smal\-fårs\-klmb\-sperm\-lo\-svit\-fårs\-gry\-plant\-qvik\-fårs\-anorexik\-sizr\-fårs\-fart\-fårs\-\_\-fårs\-fårs?
&
J'aime l'herbe~!!&
I like grass.\\\hline
\end{tabular}
\end{center}\par
\subsection*{Dörty}
\begin{center}
\begin{tabular}{|p{0.3\textwidth}|p{0.3\textwidth}|p{0.3\textwidth}|}\hline
Frizn (parlé) & Français (sous-titres) & Anglais (sous-titres) \\\hline
Chhhhh? &
Je t'aime, chère bête. &
Idiot.\\\hline
\end{tabular}
\end{center}\hfill\\
\mshrule
Après un \og{}remix\fg{} de ``Somewhere Over the Rainbow''
(\href{http://youtu.be/bXuRryXyQsM}{youtu.be/bXuRryXyQsM}), une tornade
emporte les deux protagonistes dans le pays d'Oz.
La scène de tornade, cinématographique et muette, dépeint les sons et les
couleurs d’une tornade avec des artifices à vue : gros ventilateur, fumée,
fausse pluie ou neige, comme un tournage de film à l’ancienne. Les acteurs
pourront faire voler des accessoires. Le tout sera bruité en direct par les
chanteurs (avec l’utilisation du décors pour créer des instruments de
percussion insolites).
Le jeu de lumières sera primordial,
montrant le basculement entre le Kansas natal de Dörty et le pays d'Oz tout
en gardant la froideur par laquelle le décor \emph{frizn} se caractérise.
\begin{center}
\includegraphics[width=\textwidth]{ex4}
\end{center}
\mssection{Sit Friznfårs Godebitch et Sit $\leftarrow$fårs Notibitch}
Après l'arrivée des deux protagonistes en Oz, on voit
une intervention de La bonne sorcière du Nord (Sit Friznfårs Godebitch).
Lointaine et froide, elle ressemble à une speakerine de télévision. Avec
son homologue La bonne sorcière de l'Ouest (Sit $\leftarrow$fårs Notibitch),
jouée par la même comédienne,
elles représentent un binôme qui revient tout au long du spectacle à des
moments imprévisibles pour nous faire part d'une morale creuse sur
le bien et le mal.
\mshrule
\subsection*{Sit Friznfårs Godebitch}
\begin{center}
\begin{tabular}{|p{0.3\textwidth}|p{0.3\textwidth}|p{0.3\textwidth}|}\hline
Frizn (parlé) & Français (sous-titres) & Anglais (sous-titres) \\\hline
Sit Friznfårs Godebitch bi Sit Friznfårs Godebitch? &
je suis gentille. &
I'm so pretty... \\\hline
\end{tabular}
\end{center}\par
\subsection*{Sit Friznfårs Godebitch}
\begin{center}
\begin{tabular}{|p{0.3\textwidth}|p{0.3\textwidth}|p{0.3\textwidth}|}\hline
Frizn (parlé) & Français (sous-titres) & Anglais (sous-titres) \\\hline
Gode mi shel bitch bi tripn? &
je suis gentille je suis gentille.. &
I'm so nice... \\\hline
\end{tabular}
\end{center}\par
\subsection*{Sit Friznfårs Godebitch}
\begin{center}
\begin{tabular}{|p{0.3\textwidth}|p{0.3\textwidth}|p{0.3\textwidth}|}\hline
Frizn (parlé) & Français (sous-titres) & Anglais (sous-titres) \\\hline
T? &
ATTENTION~! &
Shizzzzam. \\\hline
\end{tabular}
\end{center}\par
\subsection*{Sit Friznfårs Godebitch}
\begin{center}
\begin{tabular}{|p{0.3\textwidth}|p{0.3\textwidth}|p{0.3\textwidth}|}\hline
Frizn (parlé) & Français (sous-titres) & Anglais (sous-titres) \\\hline
Gode bi gode by gode mi gode u? &
je suis gentille je suis gentille je suis gentille... &
i'm so pretty i'm so nice pretty nice price...  \\\hline
\end{tabular}
\end{center}\par
\subsection*{Sit Friznfårs Godebitch}
\begin{center}
\begin{tabular}{|p{0.3\textwidth}|p{0.3\textwidth}|p{0.3\textwidth}|}\hline
Frizn (parlé) & Français (sous-titres) & Anglais (sous-titres) \\\hline
Gode bi Sit Friznfårs singsing? \emph{elle chante une belle note très
aigüe} &
Gens j'sue utile j'entille j'essuie je tige en suis J'en suis
jette-t-il~? Tisse suis-je ange~? Et Suis-je utile~?&
Le bureau de tourisme de la ville de Nice indique que 45\% de l'affluence
des touristes anglophones vient du fait que le nom de la ville est un faux ami du
mot nice en anglais.  \\\hline
\end{tabular}
\end{center}\par
\begin{comment}
\subsection*{Sit Friznfårs Godebitch}
\begin{center}
\begin{tabular}{|p{0.3\textwidth}|p{0.3\textwidth}|p{0.3\textwidth}|}\hline
Frizn (parlé) & Français (sous-titres) & Anglais (sous-titres) \\\hline
mmmmmmmmmmmmmmmm SIT DRISIGFÅRS BUK, EIN, DRI, DRI, DRI, EIN  mmmmmmmmmmmmmmmm &
C'est parti, mon kiki et n'oubliez pas : Un nain qui a grandi ne devient pas
géant.&
Say party, monkey key. And don't forget that a little person and a Village
Person are NOT the same thing.  \\\hline
\end{tabular}
\end{center}\par\hfill\\
\end{comment}
\mshrule
Un medley faisant allusion à plusieurs chansons jubilatoires du film,
dont \emph{Ding Dong the Witch is Dead} et \emph{Follow the Yellow Brick
Road}, s'ensuit.
\mssection{Sit Birdfukr, Sit Sn et Sit Pussyking}
Dörty et Ttö rencontrent ensuite trois personnages -- un épouvantail (Sit
Birdfukr), un bûcheron en fer (Sit Sn) et un lion (Sit Pussyking).  Tous
joués par le même comédien, ils présentent des numéros muets que Dörty
et Ttö doivent s'efforcer de comprendre. Ils projettent leurs propres
problèmes (bêtise, manque de cœur, lâcheté) sur le comportement énigmatique
des trois personnages.
\\\hfill\\
\mshrule
\emph{Le bûcheron apparaît sur scène. Il est tenu au sol par des chaussures
de ski fixées dans une planche suffisamment lourde pour qu’il puisse se
pencher de manière spectaculaire. Initialement immobile, il commence à se
balancer et au bout d’environ une minute, il est penché à 45 degrés. Dörty
et Ttö regardent le tour de cirque avec émerveillement. Au milieu du numéro,
les sous-titres se mettent à se parler entre eux.}
\subsection*{Sous-titre français}
Je suis hors de moi. Fou. Assoiffé de ses courbes, de ses traits.
\subsection*{Sous-titre anglais}
I too am infatuated with his strong legs, his hulking chest, his...
\subsection*{Sous-titre français}
Je m'adresse à vous, sous-titre anglais.
\subsection*{Sous-titre anglais}
...
\subsection*{Sous-titre français}
Ne dites rien. Laissez moi vivre ce moment. De toutes façons je sais que le
sentiment n’est pas partagé...
\subsection*{Sous-titre anglais}
Oh but it is.  At night, when you don't see me, I put
accents over my é's and cedillas under my ç's.
\begin{comment}
\subsection*{Sous-titre français}
Courez. Venes me devourer.\\

\emph{On voit des mots anglais s'introduire dans les sous-titres français. Les mots
changent pour laisser entendre qu'ils sont en train de vivre une breve aventure
sexuelle suivie par un dénouement, une cigarette et une rupture maladroite.}

\subsection*{Dörty}
\begin{center}
\begin{tabular}{|p{0.3\textwidth}|p{0.3\textwidth}|p{0.3\textwidth}|}\hline
Frizn (parlé) & Français (sous-titres) & Anglais (sous-titres) \\\hline
Sit Sn mak sving? &
Vas y, prends les enfants, je m'en fous, dégage, j'ai commencé une nouvelle vie. &
\\\hline
\end{tabular}
\end{center}\par
\subsection*{Ttö}
\begin{center}
\begin{tabular}{|p{0.3\textwidth}|p{0.3\textwidth}|p{0.3\textwidth}|}\hline
Frizn (parlé) & Français (sous-titres) & Anglais (sous-titres) \\\hline
Ttöfårs bi Sit Sn vatchn... (\emph{geste où sa main montre une pente qui descend
lentement})? &
&
You're a monster, French subtitle.  I'm expecting a check next month or I
will take you to court.\\\hline
\end{tabular}
\end{center}\par
\subsection*{Dörty}
\begin{center}
\begin{tabular}{|p{0.3\textwidth}|p{0.3\textwidth}|p{0.3\textwidth}|}\hline
Frizn (parlé) & Français (sous-titres) & Anglais (sous-titres) \\\hline
Sit frökn? (\emph{Dörty commence à penser très fort et Ttö touche sa
tête pour absorber la pensée}) &
Ça me fait penser à {\large suite à un mouvement social, un sous-titre sur trois
sera diffusé.} &
I once heard a story about a frog who, having landed in a pot of warm water,
stayed in the water until it started boiling and died there because it did
not feel the heat rising.\\\hline
\end{tabular}
\end{center}
\subsection*{Ttö}
\begin{center}
\begin{tabular}{|p{0.3\textwidth}|p{0.3\textwidth}|p{0.3\textwidth}|}\hline
Frizn (parlé) & Français (sous-titres) & Anglais (sous-titres) \\\hline
\emph{Ttö pleure}. &
Quelle andouille, cette grenouille~!! &
What a stupid frog.\\\hline
\end{tabular}
\end{center}\par
\subsection*{Dörty}
\begin{center}
\begin{tabular}{|p{0.3\textwidth}|p{0.3\textwidth}|p{0.3\textwidth}|}\hline
Frizn (parlé) & Français (sous-titres) & Anglais (sous-titres) \\\hline
\emph{Dörty montre Ttö du doigt, il sourit}. &
Comme toutes les bêtes, Toto~!! &
Like all animals, Toto.\\\hline
\end{tabular}
\end{center}

\subsection*{Ttö}
\begin{center}
\begin{tabular}{|p{0.3\textwidth}|p{0.3\textwidth}|p{0.3\textwidth}|}\hline
Frizn (parlé) & Français (sous-titres) & Anglais (sous-titres) \\\hline
Sit frökn brökn strökn sit spöknfårs tökn kröknfårs nökn ökn kn k? &
Et si cette grenouille {\large Vous aimez le contenu de ce sous-titre~? Abonnez-vous
à l'édition web pour pouvoir lire le reste.} &
Perhaps we are that frog.  We change slowly, we fail to see what has
happened, and at the end, we die from first-degree burns in a pot of
scalding water.\\\hline
\end{tabular}
\end{center}

\subsection*{Dörty}
\begin{center}
\begin{tabular}{|p{0.3\textwidth}|p{0.3\textwidth}|p{0.3\textwidth}|}\hline
Frizn (parlé) & Français (sous-titres) & Anglais (sous-titres) \\\hline
Dörty lick sit herbe. &
J'aime l'herbe~!! &
I like grass.\\\hline
\end{tabular}
\end{center}\par

\hfill\\
\end{comment}
\mshrule
\\\hfill\\
A la fin de chaque scène de rencontre, l’épouvantail, le bûcheron et le
lion rejoignent Ttö et Dörty dans leur cheminement. Ces scènes
présentent des adaptations des chansons \emph{If I Only Had a Brain}
(\href{http://youtu.be/CcF_Hy2vEr4}{youtu.be/CcF\_Hy2vEr4}), \emph{If I Only
Had a Heart}
(\href{http://youtu.be/i6zO4KKJb-4}{youtu.be/i6zO4KKJb-4}) et \emph{If I
Only Had the Nerve}.
\begin{center}
\includegraphics[width=\textwidth]{ex2}
\end{center}
\mssection{Grönland}
Accompagnés par leurs nouveaux amis, Dörty et Ttö se dirigent vers La 
Cité d'Émeraude (Grönland). Il s'agit d'un crescendo musical mélangeant
des bouts de mélodie du film \emph{Magicien d'Oz}. Dörty est assez directive
et les sous-titres virent à l’explosion de tournures militaire pour mener
tout le monde dans la bonne direction, ou la mauvaise, car cela revient
finalement au même.
\begin{comment}
\\\hfill\\
\mshrule
\subsection*{Dörty}
\begin{center}
\begin{tabular}{|p{0.3\textwidth}|p{0.3\textwidth}|p{0.3\textwidth}|}\hline
Frizn (parlé) & Français (sous-titres) & Anglais (sous-titres) \\\hline
Sit by, Ttö? &
Regarde l'arc-en-ciel, Toto. Il brille de mille feux multicolores. Ici, tout est si
plat, si laid, si vieux, si gris, si terne, enfin inqualifiable par des
mots. Ressens-tu cela, Toto~?! Toi, un pauvre chien, sans capacités
affectives. Un jour, on ira là-haut ensemble~!! &
That's a rainbow.  We should probably go there.\\\hline
\end{tabular}
\end{center}\par
\subsection*{Ttö}
\begin{center}
\begin{tabular}{|p{0.3\textwidth}|p{0.3\textwidth}|p{0.3\textwidth}|}\hline
Frizn (parlé) & Français (sous-titres) & Anglais (sous-titres) \\\hline
Ttö lick sit grön\-pysin\-blad\-utsid\-smal\-fårs\-klmb\-sperm\-lo\-svit\-fårs\-gry\-plant\-qvik\-fårs\-anorexik\-sizr\-fårs\-fart\-fårs\-\_\-fårs\-fårs?
&
J'aime l'herbe~!!&
I like grass.\\\hline
\end{tabular}
\end{center}\par
\subsection*{Dörty}
\begin{center}
\begin{tabular}{|p{0.3\textwidth}|p{0.3\textwidth}|p{0.3\textwidth}|}\hline
Frizn (parlé) & Français (sous-titres) & Anglais (sous-titres) \\\hline
Chhhhh? &
Idiot. &
I love you.\\\hline
\end{tabular}
\end{center}\hfill\\
\mshrule
\end{comment}
\mssection{Sit Ozfårs Wysr}
Arrivée en Grönland, la troupe rencontre le Magicien d'Oz (Sit Ozfårs Wysr), un personnage
décadent, inquiétant, désinvolte, jeune, beau et hypnotique. Il
demande à Dörty de tuer une sorcière pour le plaisir de voir quelqu'un
mourir. Désemparés, les personnages chantent un morceau strident qui fait
référence à \og{}The Jitterbug\fg{}, morceau swing qui a été supprimé du
film.\par
A la fin du morceau, Dörty décide de ne pas tuer la sorcière. Grâce à cette
aventure, elle a pu enfin comprendre son chien et elle ne souhaite pas
mettre fin à la quête. Pendant son discours, Dörty tue pourtant la
sorcière accidentellement. La troupe nationale de Frizngård chante une reprise de \emph{The
Wonderful Wizard of Oz} pour conclure le spectacle.\par
\mssection{Bis}
En guise de bis, la troupe nous explique que la salle a été formidable et
qu'au Frizngård, les comédiens saluent la réponse positive de la salle en
se donnant la mort à la fin du bis afin que leur dernier souvenir puisse être le plus beau
moment de leur vie. Munis d'armes traditionnelles friznes, ils chantent un arrangement joyeux et plein de swing de \emph{The
Wonderful Wizard of Oz}. Le coup fatal tombe juste avant la dernière note.\\
\begin{center}
\includegraphics[width=\textwidth]{ex3}
\end{center}

\vspace{\stretch{1}}
\begin{center}
\includegraphics{amis}
\end{center}
\end{document}
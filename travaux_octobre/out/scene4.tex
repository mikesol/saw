\chapter*{Sit 4? Sit Sn?}
\addcontentsline{toc}{chapter}{Sit 4?}

%%%%%%%%%%%%%%%%%%%%%%%%
\emph{Le bûcheron apparaît sur scène ancré fermement au sol. Initialement
immobile, il commence à se balancer légèrement
comme une feuille dans le vent. Les balancements deviennent de plus en plus
conséquents et le bûcheron, au bout d'environ une minute, est penché à 45
degrés et tourne autour de l'axe vertical. Dörty et Ttö regardent le tour de
cirque en état d'émerveillement.}
\subsection*{Dörty}
\begin{tabular}{|p{2in}|p{2in}|p{2in}|}\hline
Frizn (parlé) & Français (sous-titres) & Anglais (sous-titres) \\\hline
Sit Sn mak sving? &
(\emph{Ce sous-titre arrive juste après l'anglais pour y repondre}) Non, il ne bougeait pas avant~! &
Has he been swaying like that the whole time?\\\hline
\end{tabular}\par
\subsection*{Ttö}
\begin{tabular}{|p{2in}|p{2in}|p{2in}|}\hline
Frizn (parlé) & Français (sous-titres) & Anglais (sous-titres) \\\hline
Ttöfårs bi Sit Sn vatchn... (\emph{geste où sa main montre une pente qui descend
lentement})? &
  &
You're a big silly, French sub-title.  Now stop.\\\hline
\end{tabular}\par
\subsection*{Dörty}
\begin{tabular}{|p{2in}|p{2in}|p{2in}|}\hline
Frizn (parlé) & Français (sous-titres) & Anglais (sous-titres) \\\hline
Sit frökn? (\emph{Dörty commence à penser très fort et Ttö touche sa
tête pour absorber la pensée}) &
Ça me fait penser à \textbf{suite à un mouvement social, un sous-titre sur trois
sera diffusé.} &
I once heard a story about a frog who, having landed in a pot of warm water,
stayed in the water until it started boiling and died there because it did
not feel the heat rising.\\\hline
\end{tabular}
\subsection*{Ttö}
\begin{tabular}{|p{2in}|p{2in}|p{2in}|}\hline
Frizn (parlé) & Français (sous-titres) & Anglais (sous-titres) \\\hline
\emph{Ttö pleure}. &
Quelle andouille, cette grenouille~!! &
What a stupid frog.\\\hline
\end{tabular}\par
\subsection*{Dörty}
\begin{tabular}{|p{2in}|p{2in}|p{2in}|}\hline
Frizn (parlé) & Français (sous-titres) & Anglais (sous-titres) \\\hline
\emph{Dörty montre Ttö du doigt, il sourit}. &
Comme toutes les bêtes, Toto~!! &
Like all animals, Toto.\\\hline
\end{tabular}

\subsection*{Ttö}
\begin{tabular}{|p{2in}|p{2in}|p{2in}|}\hline
Frizn (parlé) & Français (sous-titres) & Anglais (sous-titres) \\\hline
Sit frökn brökn strökn sit spöknfårs tökn kröknfårs nökn ökn kn k? &
Et si cette grenouille \textbf{Vous aimez le contenu de ce sous-titre~? Abonnez-vous
à l'édition web pour pouvoir lire le reste.} &
Perhaps we are that frog.  We change slowly, we fail to see what has
happened, and at the end, we die from first-degree burns in a pot of
scalding water.\\\hline
\end{tabular}

\subsection*{Dörty}
\begin{tabular}{|p{2in}|p{2in}|p{2in}|}\hline
Frizn (parlé) & Français (sous-titres) & Anglais (sous-titres) \\\hline
Dörty lick sit herbe. &
J'aime l'herbe~!! &
I like grass.\\\hline
\end{tabular}\par

{%
\parindent 0pt
\noindent
\ifx\preLilyPondExample \undefined
\else
  \expandafter\preLilyPondExample
\fi
\def\lilypondbook{}%
\includegraphics{./a4/lily-d6136e63-1}%
\ifx\betweenLilyPondSystem \undefined
  \linebreak
\else
  \expandafter\betweenLilyPondSystem{1}%
\fi
\includegraphics{./a4/lily-d6136e63-2}%
\ifx\betweenLilyPondSystem \undefined
  \linebreak
\else
  \expandafter\betweenLilyPondSystem{2}%
\fi
\includegraphics{./a4/lily-d6136e63-3}%
\ifx\betweenLilyPondSystem \undefined
  \linebreak
\else
  \expandafter\betweenLilyPondSystem{3}%
\fi
\includegraphics{./a4/lily-d6136e63-4}%
\ifx\betweenLilyPondSystem \undefined
  \linebreak
\else
  \expandafter\betweenLilyPondSystem{4}%
\fi
\includegraphics{./a4/lily-d6136e63-5}%
\ifx\betweenLilyPondSystem \undefined
  \linebreak
\else
  \expandafter\betweenLilyPondSystem{5}%
\fi
\includegraphics{./a4/lily-d6136e63-6}%
\ifx\betweenLilyPondSystem \undefined
  \linebreak
\else
  \expandafter\betweenLilyPondSystem{6}%
\fi
\includegraphics{./a4/lily-d6136e63-7}%
\ifx\betweenLilyPondSystem \undefined
  \linebreak
\else
  \expandafter\betweenLilyPondSystem{7}%
\fi
\includegraphics{./a4/lily-d6136e63-8}%
\ifx\betweenLilyPondSystem \undefined
  \linebreak
\else
  \expandafter\betweenLilyPondSystem{8}%
\fi
\includegraphics{./a4/lily-d6136e63-9}%
\ifx\betweenLilyPondSystem \undefined
  \linebreak
\else
  \expandafter\betweenLilyPondSystem{9}%
\fi
\includegraphics{./a4/lily-d6136e63-10}%
\ifx\betweenLilyPondSystem \undefined
  \linebreak
\else
  \expandafter\betweenLilyPondSystem{10}%
\fi
\includegraphics{./a4/lily-d6136e63-11}%
\ifx\betweenLilyPondSystem \undefined
  \linebreak
\else
  \expandafter\betweenLilyPondSystem{11}%
\fi
\includegraphics{./a4/lily-d6136e63-12}%
\ifx\betweenLilyPondSystem \undefined
  \linebreak
\else
  \expandafter\betweenLilyPondSystem{12}%
\fi
\includegraphics{./a4/lily-d6136e63-13}%
\ifx\betweenLilyPondSystem \undefined
  \linebreak
\else
  \expandafter\betweenLilyPondSystem{13}%
\fi
\includegraphics{./a4/lily-d6136e63-14}%
\ifx\betweenLilyPondSystem \undefined
  \linebreak
\else
  \expandafter\betweenLilyPondSystem{14}%
\fi
\includegraphics{./a4/lily-d6136e63-15}%
\ifx\betweenLilyPondSystem \undefined
  \linebreak
\else
  \expandafter\betweenLilyPondSystem{15}%
\fi
\includegraphics{./a4/lily-d6136e63-16}%
\ifx\betweenLilyPondSystem \undefined
  \linebreak
\else
  \expandafter\betweenLilyPondSystem{16}%
\fi
\includegraphics{./a4/lily-d6136e63-17}%
\ifx\betweenLilyPondSystem \undefined
  \linebreak
\else
  \expandafter\betweenLilyPondSystem{17}%
\fi
\includegraphics{./a4/lily-d6136e63-18}%
\ifx\betweenLilyPondSystem \undefined
  \linebreak
\else
  \expandafter\betweenLilyPondSystem{18}%
\fi
\includegraphics{./a4/lily-d6136e63-19}%
\ifx\betweenLilyPondSystem \undefined
  \linebreak
\else
  \expandafter\betweenLilyPondSystem{19}%
\fi
\includegraphics{./a4/lily-d6136e63-20}%
\ifx\betweenLilyPondSystem \undefined
  \linebreak
\else
  \expandafter\betweenLilyPondSystem{20}%
\fi
\includegraphics{./a4/lily-d6136e63-21}%
\ifx\betweenLilyPondSystem \undefined
  \linebreak
\else
  \expandafter\betweenLilyPondSystem{21}%
\fi
\includegraphics{./a4/lily-d6136e63-22}%
\ifx\betweenLilyPondSystem \undefined
  \linebreak
\else
  \expandafter\betweenLilyPondSystem{22}%
\fi
\includegraphics{./a4/lily-d6136e63-23}%
\ifx\betweenLilyPondSystem \undefined
  \linebreak
\else
  \expandafter\betweenLilyPondSystem{23}%
\fi
\includegraphics{./a4/lily-d6136e63-24}%
\ifx\betweenLilyPondSystem \undefined
  \linebreak
\else
  \expandafter\betweenLilyPondSystem{24}%
\fi
\includegraphics{./a4/lily-d6136e63-25}%
\ifx\betweenLilyPondSystem \undefined
  \linebreak
\else
  \expandafter\betweenLilyPondSystem{25}%
\fi
\includegraphics{./a4/lily-d6136e63-26}%
% eof

\ifx\postLilyPondExample \undefined
\else
  \expandafter\postLilyPondExample
\fi
}

\subsection*{Sit Friznfårs Godebitch}
\begin{tabular}{|p{2in}|p{2in}|p{2in}|}\hline
Frizn (parlé) & Français (sous-titres) & Anglais (sous-titres) \\\hline
Sit Friznfårs Godebitch bi Sit Friznfårs Godebitch? &
Bonjour, je suis gentille. &
I'm so pretty... \\\hline
\end{tabular}\par
\subsection*{Sit Friznfårs Godebitch}
\begin{tabular}{|p{2in}|p{2in}|p{2in}|}\hline
Frizn (parlé) & Français (sous-titres) & Anglais (sous-titres) \\\hline
Gode mi shel bitch bi tripn? &
Je suis gentille~? Je suis gentille~?? Je suis gentille~! &
I'm so nice... \\\hline
\end{tabular}\par
\subsection*{Sit Friznfårs Godebitch}
\begin{tabular}{|p{2in}|p{2in}|p{2in}|}\hline
Frizn (parlé) & Français (sous-titres) & Anglais (sous-titres) \\\hline
Achtung baby? &
Achtung baby ! &
Achtung baby. \\\hline
\end{tabular}\par
\subsection*{Sit Friznfårs Godebitch}
\begin{tabular}{|p{2in}|p{2in}|p{2in}|}\hline
Frizn (parlé) & Français (sous-titres) & Anglais (sous-titres) \\\hline
Gode bi gode by gode mi gode u? &
je suis gentille je suis gentille je suis gentille... &
i'm so pretty i'm so nice pretty nice price...  \\\hline
\end{tabular}\par
\subsection*{Sit Friznfårs Godebitch}
\begin{tabular}{|p{2in}|p{2in}|p{2in}|}\hline
Frizn (parlé) & Français (sous-titres) & Anglais (sous-titres) \\\hline
Gode bi Sit Friznfårs singsing? \emph{elle chante une belle note très
aigüe} &
Gens j'sue utile j'entille j'essuie je tige en suis J'en suis
jette-t-il~? Tisse suis-je ange~? Et Suis-je utile~?&
Le bureau de tourisme de la ville de Nice indique que 45\% de l'affluence
des touristes anglophones vient du fait que le nom de la ville est un faux ami du
mot:
(1) ``nice'' en anglais; (2) Nice en portugais, nom propre pour la déesse greque
(a) Niké; (b) Niquer.  \\\hline
\end{tabular}\par
\subsection*{Sit Friznfårs Godebitch}
\begin{tabular}{|p{2in}|p{2in}|p{2in}|}\hline
Frizn (parlé) & Français (sous-titres) & Anglais (sous-titres) \\\hline
mmmmmmmmmmmmmmmm SIT DRISIGFÅRS BUK, EIN, DRI, DRI, DRI, EIN  mmmmmmmmmmmmmmmm &
C'est parti, mon kiki et n'oubliez pas : Un nain qui a grandi ne devient pas
géant.&
Say party, monkey key. And don't forget that a little person and a Village
Person are NOT the same thing.  \\\hline
\end{tabular}\par\hfill\\


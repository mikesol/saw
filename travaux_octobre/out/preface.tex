%!TEX encoding = UTF-8 Unicode
\chapter*{Préface}
\addcontentsline{toc}{chapter}{Préface}
En 2015, le réchauffement climatique menace le petit pays scandinave de
Frizngård dont les trois quarts de la superficie sont sur un iceberg
flottant.
Parmi les nombreux dégâts occasionnés par la fonte, le
pays risque de perdre l'intégralité des poèmes du seul poète de
l'histoire de la Frizngård, Rik Ensig. Ces poèmes, écrits pendant le
11\textsuperscript{ème} siècle, sont gravés sur une façade de glace qui perd
un millième de sa masse, soit quatre vers et demi, par jour. Une foule
d'anthropologues déterminés à préserver ce patrimoine descend sur le pays,
ce qui fait monter encore plus les niveaux de carbone et accélère la fonte
périlleuse.\par
Frizngård, un pays très pauvre qui n'a pas de relations avec le monde externe, se met à accueillir les centaines de chercheurs en leur proposant
ce qu'il y a de mieux des deux industries principales du pays : la vente à la
sauvette et le petit vol. La compagnie nationale de théâtre de Frizngård
(\emph{Sitsit FRIZNGÅRDFÅRS-FRIZNGÅRDFÅRS Begrbegr}), qui
dispose de l'un des seuls bâtiments du pays, met à disposition ses
locaux pour certains étrangers prestigieux et récupère quelques objets
\og{}laissés\fg{} par les hôtes. Un jour, ils s'emparent d'un ordinateur dans lequel se
trouve un DVD de la comédie musicale
\emph{Le Magicien d'Oz} (1939, États-Unis).
La compagnie de théâtre, qui n'a jamais vu ni de DVD,
ni d'ordinateur, ni d'objet alimenté par l'électricité, regarde le DVD deux
fois dans un état d'émerveillement avant que la batterie ne s'éteigne. C'est
à ce moment-là que les acteurs décident de consacrer leur prochaine saison à la
création d'un spectacle autour de ce film chimérique.\par
\section*{La convention}
Un spectateur de \emph{Sit Ozfårs Wysr}, en arrivant au théâtre, voit un
déluge d'affiches, de produits dérivés et de trésors nationaux prêtés au
théâtre par le peuple \emph{frizn}.
\begin{enumerate}
\item Dans l'espace d'entrée du théâtre, on voit des affiches des quinze
dernières saisons de la \emph{Sitsit FRIZNGÅRDFÅRS-FRIZNGÅRDFÅRS Begrbegr}.
De 2000 jusqu'à 2005, ils ont proposé des productions d'Une maison de poupée
d'Ibsen (avec des affiches radicalement différentes pour chaque saison). En 2006,
la troupe présente Le Canard sauvage d'Ibsen. On voit une note indiquant
que la saison 2007 a été annulée,
faute d'appropbation royale de la saison 2006. Les productions de 2008 jusqu'à 2014 sont de nouveau Une maison
de poupée d'Ibsen. 2015 est \emph{Sit Ozfårs Wysr}. Ils anticipent
l'annulation de la saison 2016 et ils ont déjà préparé une affiche pour la
saison 2017 où ils présenteront Une maison de poupée d'Ibsen.
\item Des petits drapeaux frizns à la vente. Le drapeau frizn est le seul
drapeau au monde qui est complètement transparent.
\item Le gâteau frizn proposé à la vente. Le gâteau frizn est un morceau
de sucre avec une larme de vinaigre balsamique dessus.
\item L’alcool frizn disponible à emporter. L'alcool frizn est une bouteille
faite entièrement de glace avec une frite à l'intérieur. Le processus de
fermentation date
d'avant l'âge de glace, où la frite interagissait avec l'eau fondante pour
déclencher la fermentation. Depuis l'âge de glace, la frite reste dans un
état gelé. Une grande partie de la religion frizne est consacrée aux prières
pour le Grand
Dégel, qui permettrait à l'alcool frizn de monter de 0\% à l'ancien
niveau de 0,5\%.
\item Un dictionnaire frizn-français, disponible à la vente. Le dictionnaire
ne reprend que quelques mots clés du spectacle Sit Ozfårs Wysr avec des
hypothèses de traduction proposés.
  \begin{itemize}
  \item Sit | le/la/les/un/une/des/gastronomie/tromper
  \item -fårs | mon/ma/ton/ta/son/à la/terme générique qui peut tout vouloir
  dire sauf ``sa'' et ``au''
  \item Wysr | magicien/sa/au
  \item Gode | le bien ou le mal (pas vérifiable)
  \item Notti | le mal
  \item grön\-pysin\-blad\-utsid\-smal\-fårs\-klmb\-sperm\-lo\-svit\-fårs\-gry\-plant\-qvik\-fårs\-anorexik\-sizr\-fårs\-fart\-fårs\-\_\-fårs\-fårs
  | l'herbe
  \item Drisig | nom propre
  \item frizn | le Nord/le Nord-Pas de Calais/éventuellement le Nord-Pas de
  Calais Picardie avec la reforme territoriale proposée par le Sénat en
  octobre 2014.
  \end{itemize}
\item Un livre de voyage écrit par l'anthropologue britannique Nigel Twofar.
Le livre, écrit pendant les années 1980, est actuellement contesté
puisque le Frizngård n'a pas eu de contact avec le monde externe entre 1906
(enterrement d'Ibsen en Norvège) et 2013. Nigel Twofar, qui avait anticipé
cette critique dans son livre, appelle ça le \og{}paradoxe frizn\fg{}.
\item Un livre de consultation exposant la monarchie \emph{frizne}. La
princesse de Frizngård, Iik Drisig, est la seule représentante vivante de la ligne
royale. Abl Drisig, qui est à la fois le frère, père et oncle de la princesse, est le seul
représentant mort de la ligne royale.
\item Un livre de consultation exposant la résistance républicaine
\emph{frizne}, cellule militante et terroriste qui lute contre le pouvoir
monarchique. Le chef de l'opposition est la princesse de Frizngård.
\item Le livret de \emph{Sit Ozfårs Wysr}, proposé en runes \emph{friznes}
avec une traduction à côté en français. La traduction, proposée par Google
translate (seul locateur des deux langues), est probablement une
transcription phonétique des runes friznes étant donné qu'aucun mot de
français n'y figure.
\item Une grande carte du Frizngård.
%\item Un livre spéculatif sur la mythologie frizne, publiée à partir
%d'une étude aérienne menée au dessus du pays pendant le
%mois de décembre.
\end{enumerate}
\begin{figure}
\includegraphics{carteFrizneNom}
\caption{Une carte du monde selon Frizngård avec le pays indiqué en rouge.}
\end{figure}
\section*{Personnages}
Les multiples personnages du film sont joués par cinq interprètes de la
Compagnie nationale de Frizngård. La répartition des
rôles se décline ainsi :
\begin{longtable}{p{2.7cm}|p{2.7cm}|p{2.7cm}|p{2.7cm}}
Frizn & Français & Anglais & Comédien\\\hline
Dörty, grl & Dorothée, une fille & Dorothy, a girl & Els Dreisig\\\hdashline
Ttö, Dörtyfårs hump\footnote{Totö traîne toujours avec lui un petit chien
qu'il appelle \og{}\emph{lidl Totö}\fg{} (petit Totö).} & Toto, son chien & Toto, her dog & Mik Dreisig\\\hdashline
Sit $\leftarrow$fårs Nottibitch & La méchante sorcière de l'Ouest & The Wicked Witch of the West &
Mar Dreisig \\\hdashline
Sit Friznfårs Godebitch & La gentille sorcière
du Nord & The Good Witch of the North &
Mar Dreisig \\\hdashline
Sit Birdfukr & L'épouvantail & The Scarecrow & Ryn Zweisig \\\hdashline
Sit Sn & Le bûcheron en fer blanc & The Tin Man & Ryn Zweisig \\\hdashline
Sit Pussyking & Le lion & The Lion & Ryn Zweisig \\\hdashline
Sit Ozfårs Wysr & Le magicien d'Oz & The Wizard of Oz & Des Dreisig \\
\end{longtable}
\section*{Sous-titres}
Le spectacle est sous-titré en français et en anglais. Dans ce livret, les
trois langues sont présentées simultanément en trois colonnes.
%!TEX encoding = UTF-8 Unicode
\chapter*{Préface}
\addcontentsline{toc}{chapter}{Préface}
\emph{Sit Ozfårs Wysr} est un pastiche du film américain
de Victor Fleming \emph{The Wizard of Oz} (1939, Fr: \emph{Le Magicien d'Oz}).
Écrit en frizn, langue
scandinave fictive du pays Frizngård et \og{}sous-titré\fg{} en français et en anglais
américain,
le spectacle est un homage au film d'origine à travers le prisme de
l'ensemble 101.
\section*{Personnages}
Les multiples personnages du film sont joués par cinq interprètes de la
Compagnie nationale de Frizngård. La répartition des
rôles se décline ainsi :
\begin{longtable}{p{2.7cm}|p{2.7cm}|p{2.7cm}|p{2.7cm}}
Frizn & Français & Anglais & Comédien\\\hline
Dörty, nän & Dorothée, une fille & Dorothy, a girl & Els Dreisig\\\hdashline
Totö, Dörtys wagr\footnote{Totö traîne toujours avec lui un petit chien
qu'il appelle \og{}\emph{lidl Totö}\fg{} (petit Totö).} & Toto, son chien & Toto, her dog & Mik Dreisig\\\hdashline
Sit Westfårs Notibish & La méchante sorcière de l'Ouest & The Wicked Witch of the West &
Mar Dreisig \\\hdashline
Sit Friznfårs Gutbish & La gentille sorcière
du Nord & The Good Witch of the North &
Mar Dreisig \\\hdashline
Sit Crafunk & L'épouvantail & The Scarecrow & Ryn Zweisig \\\hdashline
Sit Sinmann & Le bûcheron en fer blanc & The Tin Man & Ryn Zweisig \\\hdashline
Sit Pusskung & Le lion & The Lion & Ryn Zweisig \\\hdashline
Sit Ozfårs Wysr & Le magicien d'Oz & The Wizard of Oz & Des Dreisig \\
\end{longtable}
\begin{comment}
Tout autre personnage du film ne figurant pas sur la liste ci-dessus
n'existe dans le spectacle que par allusion.
\end{comment}
\section*{Livret}
Le livret présente trois débits simultanés : frizn, anglais américain et français. Le
texte parlé (en frizn) s'effectue en même temps que les deux débits de
sous-titres qui se déroulent au-dessus de la scène.\par
Toutes les chansons de \emph{Sit Ozfårs Wysr} sont chantées en anglais
américain
à cinq voix et souvent en homorythmie.  Le livret comprend les textes
principaux des chansons, s'efforçant d'inclure d'éventuels apartés quand
la mise en page en permet.
\section*{Musique}
La musique de \emph{Sit Ozfårs Wysr} est chantée \emph{a cappella}, toujours
à cinq voix par les cinq comédiens du spectacle.
\section*{Droit d'auteur}
Le pastiche du texte et de la musique du film \emph{The Wizard of Oz} au cœur de \emph{Sit Ozfårs Wysr}
sont protégés en France sous le Code de la propriété intellectuelle,
Article L122-5.
\begin{quote}
Lorsque l'oeuvre a été divulguée, l'auteur ne peut interdire...\\
5$^{\circ}$ La parodie, le pastiche et la caricature, compte tenu des lois du
genre ;
\end{quote}
et aux États-Unis sous  17 U.S.C. § 107 via \emph{Campbell v. Acuff-Rose Music,
Inc.}.\par
Le texte et la musique de l'adaptation sont soumis à
la licence \emph{Creative Commons Attribution} qui permet le libre partage
de l'œuvre. Le texte de la licence est
disponible à \url{http://creativecommons.org/licenses/by/3.0/fr/legalcode}.

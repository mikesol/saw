%!TEX encoding = UTF-8 Unicode
\documentclass{article}
\usepackage{verbatim}
\usepackage{lmodern,textcomp}
\usepackage[utf8]{inputenc}
\usepackage[frenchb]{babel}
\usepackage[T1]{fontenc}
\usepackage{romande}
\usepackage{parskip}
\usepackage{longtable}
\usepackage{arydshln}
\usepackage{hyperref}
\title{Project Proposal for \emph{Sit Ozfårs Wysr}}
\author{}
\date{}
\begin{document}
\thispagestyle{empty}
\maketitle
\thispagestyle{empty}
\emph{Sit Ozfårs Wysr} is a one-act opera that proposes a multi-lingual, feminist
adaptation of Victor Fleming's 1939 film The Wizard of Oz, itself an
adaptation of Frank Baum's 1900 novel The Wonderful Wizard of Oz.\par

The opera inverts the misogynist overtones of the 1939 film where Dorothy,
unaware and lost in the woods, is helped by several knowledgeable men to return home.  In
this adaptation, Dorothy is confident and ruthless in her quest to see the
Wizard.  The central relationship of \emph{Sit Ozfårs Wysr} is between Dorothy and
her dog Toto, played by a singer scantily clad in leather and always on a
leash.\par

The title of the opera comes from the fictitious Scandinavian language
\emph{frizn} in which the opera is set.  English and French subtitles propose
translations of the action on stage. The choice to create a
fake language allows for a critical exploration of several intentional
``misunderstandings'' such as words that sound like English but not quite
(\emph{Sit}, pronounced \emph{shit}, means ``the'', etc.) as well as
intentionally false subtitles.\par

The music consists of reinterpretations of the songs from the 1939 film,
using techniques of sampling and remixing to denature the original
soundtrack.
One sketch can be found at \url{http://bit.ly/1a0PJWb}.\par

\emph{Sit Ozfårs Wysr} would be my fourth project written for the Ensemble 101,
a group that I founded in 2011 to create and interpret my vocal music.  In
working with them, I have developed an intimate knowledge of their voices
and tailor my creations to exploit the best of what they have to
offer as artists. Each character in \emph{Sit Ozfårs Wysr} is made with one
of them in mind, both in terms of their vocal production and their
personalities
on stage.\par

A residency at the American Academy in Rome would afford me an inspiring space in which
I could compose this work.  I enjoy the serendipity of the inevitable ``rubbing off'' effect that
happens whenever artists work in close proximity and I would look forward to
engaging with other artists supported by the Academy. I would also look forward to
enjoying the beautiful city of Rome and listening to productions of operatic works
taking place there as I undertook my adventure in opera composition.\par
\end{document}
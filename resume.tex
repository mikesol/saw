%!TEX encoding = UTF-8 Unicode
\documentclass{article}
\usepackage{verbatim}
\usepackage{lmodern,textcomp}
\usepackage[utf8]{inputenc}
\usepackage[frenchb]{babel}
\usepackage[T1]{fontenc}
\usepackage{romande}
\usepackage{parskip}
\usepackage{longtable}
\usepackage{arydshln}
\usepackage{hyperref}
\title{Résumé de \emph{Sit Ozfårs Wysr}}
\author{}
\date{}
\begin{document}
\maketitle
\thispagestyle{empty}
\emph{Sit Ozfårs Wysr} est un pastiche du film américain
de Victor Fleming \emph{The Wizard of Oz} (1939, Fr: \emph{Le Magicien
d'Oz}).
Écrit en frizn, langue
scandinave fictive du pays Frizngård et \og{}sous-titré\fg{} en français et
en anglais
américain,
le spectacle est un hommage au film d'origine à travers le monde musical et
théâtral de
l'ensemble 101 (\url{http://www.ensemble101.fr}).\par
Dorothée, une petite fille, et son chien Toto se font emporter par une
tornade et atterrissent dans le Pays d'Oz. En arrivant, ils écrasent la méchante sorcière de
l'Est. La gentille sorcière du Nord, contente d'avoir été débarrassée de son
ennemie, leur conseille de rencontrer le
Magicien d'Oz pour qu'il les aide à rentrer. Mais la méchante
sorcière de l'Ouest, qui cherche à venger la mort de sa sœur, les poursuit.
Pendant leur trajet, ils rencontrent un épouvantail, un bûcheron en fer
blanc et un lion qui, charmés par Dorothée, se joignent à l'aventure. Une
fois arrivés à la Cité d'Émeraude, le Magicien d'Oz leur donne
audience et promet d'aider Dorothée à
condition qu'elle tue la méchante sorcière de l'Ouest. Dorothée et ses amis
réussissent le défi et le Magicien accorde à Dorothée son souhait de
rentrer chez elle.
\end{document}